%%
%% starspan paper
%% Carlos A. Rueda-Velasquez
%% Jonathan E. Greenberg
%% Susan L. Ustin
%% $Id$

\documentclass{elsart}
\usepackage{natbib,url,amssymb}

\journal{Computers \& Geosciences}


\begin{document}
\begin{frontmatter}

\title{STARSPAN: 
A Tool for Fast Selective Pixel Extraction 
from Remotely Sensed Data}
\thanks[label1]{Available from \url{http://starspan.casil.ucdavis.edu}}

\author{Carlos A. Rueda\corauthref{cor}},
\corauth[cor]{Corresponding author.}
\ead{carueda@ucdavis.edu}
\author{Jonathan E. Greenberg\thanksref{greenberg}},
\thanks[greenberg]{Present address: NASA Ames Research Center...}
\ead{greenberg@ucdavis.edu}
\author{Susan L. Ustin}
\ead{slustin@ucdavis.edu}

\address{Center for Spatial Technologies and Remote Sensing (CSTARS).
The Barn. University of California, Davis. One Shields Avenue, Davis, CA 95616}

%%%%%%%%%%%%%%%%%%%%%%%%%%%%%%%%%%%%%%%%%%%%%%%%%%%%%%%%%%%%%%%%%%%%%%%%%%%
\begin{abstract}
	
	Many research endeavours in environmental and ecological
	sciences [[?]] involve the utilization of a combined GIS
	and remote sensing approach. As the amount of data is
	steadily increasing due to the availability of more
	instruments as well as improved spatio-temporal-spectral
	resolutions, the risk for bottlenecks in data management
	and processing is also becoming a more important
	consideration that must be properly addressed for timely
	processing and results. Although there exist computer
	applications to help the scientist in these endeavours,
	there is an apparent absence of specific tools to
	integrate information from both GIS and remote sensing
	sources. In this paper we describe STARSPAN, a computer
	program focused on a specific yet crucial aspect of data
	integration which is the extraction of pixel data from
	spectral raster images according to specified geometry
	features. Development of the tool started as part of a
	project for mapping invasive plant species in the
	Sacramento-San Joaquin Delta Region (approximately
	$3400$~km$^2$ in California's Central Valley) using
	field data and airborne hyperspectral imagery. STARSPAN
	is an open source project implemented in C++ and
	developed under object-oriented design principles so it
	can be extended with new functionalities. It can be
	freely downloaded from
	\url{http://starspan.casil.ucdavis.edu}. In the context
	of real examples, we explain the usage of the tool, how
	it can be extended, and discuss further directions for
	its improvement.
	
	
\end{abstract}

%%%%%%%%%%%%%%%%%%%%%%%%%%%%%%%%%%%%%%%%%%%%%%%%%%%%%%%%%%%%%%%%%%%%%%%%%%%
\begin{keyword}
GIS \sep remote sensing \sep image processing
% PACS codes here, in the form: \PACS code \sep code
\end{keyword}

\end{frontmatter}


%%%%%%%%%%%%%%%%%%%%%%%%%%%%%%%%%%%%%%%%%%%%%%%%%%%%%%%%%%%%%%%%%%%%%%%%%%%
\section{Introduction}\label{intro}


To meet evolving challenges in performing digital remote
sensing analysis, images are becoming significantly larger
and the algorithms to analyze the data are becoming much
more complex. Aerial and spaceborne sensors are generating
imagery with increasing spatial, spectral, radiometric and
temporal resolution, and multiple image datasets (used
either for multitemporal analysis or mosaicking to increase
spatial extent) are become more frequent in the literature
(CITATIONS). While computers are becoming more powerful,
image analysis software has been lagging in novel ways of
dealing with large-dataset analyses. We present a new
approach to performing classification and biophysical
parameter analysis which can greatly speed up the time it
takes to investigate multiple, potentially complex, analysis
techniques and perform validation.
	
Image analysis, broadly, typically follows three major
steps: 1) collection of training and test data, 2) perform
relevant transforms on the image data (e.g. calculation of
spectral indices or application of texture filters), 3)
application of training data to the particular analysis
(either classification algorithms, or statistical
relationships between spectral transforms and some
biophysical variable of interest), and 4) validation of the
analysis using independent test data. The ultimate goal is
often to use the best analysis algorithm selected from
multiple different approaches and produce an accurate map of
the classes or biophysical variable of interest. Training
and test data are pixels or collections of pixels (with all
associated z-profile information) of which the class or
biophysical variable of interest is known. These can be
collected through field data collection (typically in
conjunction with a GPS) or through manual selection of
pixels based on user knowledge of the site. Traditionally,
major image analysis packages (e.g. RSI ENVI, Erdas Imagine,
PCI Geomatica, GRASS GIS) require the user to perform the
most computationally limiting steps, steps 2 and 3, on the
entire image or images in order to determine if a particular
image transform and image analysis methodology will yield
sufficiently accurate results. If new transforms or analyses
are warranted, the entire image must be processed again.
With large datasets this is computationally expensive and
requires a significant amount of storage space, particularly
when image transforms are involved.
	
We make the observation that when performing image analysis,
only an extremely small fraction of pixels (usually
numbering in the thousands) are actually useful to make
decisions on accuracy, since only a small fraction would
have been visited in the field. As such, we have developed a
technique to extract pixel data from only known sites,
perform both spectral and spatial transforms on these data,
classify/determine biophysical variables, and perform
accuracy assessment. Only once the ?correct? combination of
image transforms and analysis algorithms has been determined
is it necessary to actually apply these to an entire image.

	%%%%%%%%%%

STARSPAN is an open source computer program developed at
Center of Spatial Technologies and Remote Sensing (CSTARS)
for fast and flexible extraction of pixel data from spectral
raster images. The tool was developed as part of a project
for mapping invasive plant species using hyperspectral
imagery \citep{ustin04}.
	
Main inputs for the program are one or multiple raster files
and a vector file defining a set of geometry features. The
basic operation performed by STARSPAN is the extraction of
the spectral data from the raster files whose pixels are
geometrically contained in the geometry features (points,
lines, polygons) in the vector data. Extracted pixel data
are stored in a number of different formats, including ENVI
\citep{envi} (standard image and spectral library) and CSV
(comma-separated values), a common text-based format that
can be imported into standard applications for further
analysis. Other functionalities include the generation of
band specific statistics for specified features, extraction
of features into individual rasters (mini-raster
generation), update options, and generation of paired pixel
band values for subsequent calculation of calibration
coefficients.
	
An important goal is that of software extensibility, hence
STARSPAN is being developed under object-oriented design
principles so it can be easily extended with new
functionalities deriving from the core extraction process.

	[[...]]

%%%%%%%%%%%%%%%%%%%%%%%%%%%%%%%%%%%%%%%%%%%%%%%%%%%%%%%%%%%%%%%%%%%%%%%%%%%
\section{A use case}

	[[describe use in delta project...]]

%%%%%%%%%%%%%%%%%%%%%%%%%%%%%%%%%%%%%%%%%%%%%%%%%%%%%%%%%%%%%%%%%%%%%%%%%%%
\section{Usage}

In this section we provide a detailed description of the
main operations performed by STARSPAN.

Figure \ref{fig-usage} contains the usage message given by
the program.

\begin{figure}[!ht]
\centering
\scriptsize{
\begin{verbatim}
USAGE:
  starspan <inputs/commands/options>...

   inputs:
      -raster <filenames>...
      -vector <filename>
      -speclib <filename>
      -update-csv <filename>
      -update-dbf <filename>

   commands:
      -raster_field <name>
      -raster_directory <directory>
      -csv <name>
      -envi <name>
      -envisl <name>
      -stats outfile.csv [avg|mode|stdev|min|max]...
      -calbase <link> <filename> [<stats>...]
      -report
      -dbf <name>
      -dump_geometries <filename>
      -mr <prefix>
      -jtstest <filename>

   options:
      -fields field1 field2 ... fieldn
      -pixprop <minimum-pixel-proportion>
      -noColRow
      -noXY
      -fid <FID>
      -skip_invalid_polys
      -progress [value]
      -RID_as_given
      -verbose
      -ppoly
      -in
      -srs <srs>
\end{verbatim}
}
\label{fig-usage}
\caption{Usage message}
\end{figure}

%%%%%%%%%%%%%%%%%%%%%%%%%%%%%%%%%%%%%%%%%%%%%%%%%%%%%%%%%%%%%%%%%%%%%%%%%%%
\subsection{Inputs and general options}

Raster data is given through the \verb|-raster| option:

	\verb|-raster| $R_1\ R_2\ \cdots\ R_n$

where $R_1, R_2, \ldots, R_n$ are raster datasets recognized
by the GDAL library.

Vector data is given with the \verb|-vector| option:
	
	\verb|-vector| $V$

where $V$ is a vector datasource recognized by the OGR library.

There are some general options ...

	\verb|-fields| field$_1$ field$_2\ \cdots$ field$_n$
	
Only the specified fields will be transferred from vector
file $V$ to the output. Use spaces to separate the field
names. Example:

   \verb|starspan -vector V -raster R -csv my.csv  -fields species foo|

will extract the fields with names `species' and `foo.' If
no field from the vector metadata is desired, just give the
keyword \verb|-fields| with no arguments. By default, all
fields from vector will be extracted.

	\verb|-pixprop| value

Minimum proportion of pixel area in intersection so that the
pixel is included. A value in [0.0, 1.0] must be given. Only
used in intersections resulting in polygons. By default, the
pixel proportion is 0.5.

%%%%%%%%%%%%%%%%%%%%%%%%%%%%%%%%%%%%%%%%%%%%%%%%%%%%%%%%%%%%%%%%%%%%%%%%%%%
\subsection{Statistics}

Command \verb|-stats| is used to generate basic band-wise
statistics for each intersecting feature. Example:

	\verb|starspan -vector V -raster R -stats mystats.csv avg mode|

Creates \verb|mystats.csv|, a CSV (comma-separated-values)
file containing computed statistics for each FID. In the
above example, codes \verb|avg| and \verb|mode| indicate to
compute the average and mode statistics. Available
statistics are shown in Table \ref{table-stats}.

\begin{table}[!ht]
\centering
\caption{Statistics}
\begin{tabular}{|l|l|}
\hline 
 code        & computes for each feature\\
\hline
\verb|avg  |  &   average of pixels in feature\\
\verb|mode |  &   Most frequent value\\
\verb|stdev|  &   sample standard deviation of pixels in feature\\
\verb|min  |  &   minimum value of pixels in feature\\
\verb|max  |  &   maximum value of pixels in feature\\
\hline
\end{tabular}
\label{table-stats}
\end{table}

The columns in the resulting \verb|myfile.csv| are:

	\begin{itemize}
	
		\item Feature ID
		
		\item Names of attributes from the given input
				vector $V$ associated to FID. If option
				\verb|-fields| is given, only the given fields will
				be extracted.
		
		\item Band names. Each name takes the form
				$s$\_Band\#, where $s$ is the code of the
				statistics, eg. \verb|stdev|, and \# is the band
				number according to given rasters. \end{itemize}

Subsequent lines contain corresponding values for each FID.

%%%%%%%%%%%%%%%%%%%%%%%%%%%%%%%%%%%%%%%%%%%%%%%%%%%%%%%%%%%%%%%%%%%%%%%%%%%
\section{Software requirements}

	In order to build STARSPAN the following libraries are
	required.
	
	\begin{itemize}
	
		\item GEOS, Geometry Engine \citep{geos}.
				This library is required to support geometry operations.
				
		\item GDAL, Geospatial Data Abstraction Library \citep{gdal}.
	
	\end{itemize}



%%%%%%%%%%%%%%%%%%%%%%%%%%%%%%%%%%%%%%%%%%%%%%%%%%%%%%%%%%%%%%%%%%%%%%%%%%%
\section{Conclusion and further work}

	[[...]]
	
Currently STARSPAN has been tested and used on Linux based
operating environments. On the works is to make the tool
publicly available and easy to install on different
operating systems.


For up-to-date documentation, including installation
instructions and user manual, please visit:
\url{http://starspan.casil.ucdavis.edu}

%%%%%%%%%%%%%%%%%%%%%%%%%%%%%%%%%%%%%%%%%%%%%%%%%%%%%%%%%%%%%%%%%%%%%%%%%%%
\section*{Acknowledments}

This development has been supported by California
State Department of Boating and Waterways. We acknowledge
the following people at the CSTARS laboratory for their
input and tests of the tool: Emma Underwood, Solomon
Dobrowski, Mike Whiting, Carlos Ramirez, Melinda Mulitsch,
Larry Ross, George Scheer, Quinn Hart, and Wayland Yu. The
people behind the GDAL, GEOS and JTS programming
libraries, on which STARSPAN is constructed, are also
gratefully acknowledged.



	
%%%%%%%%%%%%%%%%%%%%%%%%%%%%%%%%%%%%%%%%%%%%%%%%%%%%%%%%%%%%%%%%%%%%%%%%%%%
\bibliographystyle{elsart-harv}
\bibliography{starspan}

\end{document}


