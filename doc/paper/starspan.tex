%%
%% starspan
%% 2005-01-17
%% Carlos A. Rueda-Velasquez
%%
\documentclass{elsart}
\usepackage{natbib,url,amssymb}

\journal{Computers \& Geosciences}


\newcommand{\starspan}{STARSPAN{}}

\begin{document}
\begin{frontmatter}

\title{\starspan: 
A Tool for Fast Selective Pixel Extraction 
from Remotely Sensed Data}
\thanks[label1]{Available from \url{http://starspan.casil.ucdavis.edu}}

\author{Carlos A. Rueda\corauthref{cor}},
\corauth[cor]{Corresponding author.}
\ead{carueda@ucdavis.edu}
\author{Jonathan Greenberg\thanksref{greenberg}}
\thanks[greenberg]{Present address: NASA Ames ...}
\ead{greenberg@ucdavis.edu}

\address{Center for Spatial Technologies and Remote Sensing (CSTARS).
The Barn. University of California, Davis. One Shields Avenue, Davis, CA 95616}



% use optional labels to link authors explicitly to addresses:
% \author[label1,label2]{}
% \address[label1]{}
% \address[label2]{}


\begin{abstract}
	
	\starspan{} is a computer program for fast extraction of data from spectral
	raster images.
	
\end{abstract}

\begin{keyword}
GIS \sep remote sensing
% PACS codes here, in the form: \PACS code \sep code
\end{keyword}

\end{frontmatter}


%%%%%%%%%%%%%%%%%%%%%%%%%%%%%%%%%%%%%%%%%%%%%%%%%%%%%%%%%%%%%%%%%%%%%%%%%%%
\section{Introduction}\label{intro}

	\starspan{} is an open source computer program developed at Center of Spatial
	Technologies and Remote Sensing (CSTARS) for fast and flexible extraction of
	pixel data from spectral raster images. The tool was developed as part
	of project for mapping invasive species ... \citep{ustin04}
	It works on two basic inputs: 1.Raster
	data: one or multiple raster files 2.Vector data: a file defining a set of
	geometry features The basic operation performed by \starspan{} is the
	extraction of the spectral data from the raster files whose pixels are
	geometrically contained in the geometry features (points, lines, polygons)
	in the vector data. Extracted pixel data are stored in a number of different
	formats, including CSV files, a common text-based format that can be
	imported in many standard applications for further processing. Other
	functionalities include: band-wise statistics generation per feature,
	generation of paired pixel band values for subsequent calculation of
	calibration coefficients, update options, and generation of output data in
	ENVI formats (standard image and spectral library).


%%%%%%%%%%%%%%%%%%%%%%%%%%%%%%%%%%%%%%%%%%%%%%%%%%%%%%%%%%%%%%%%%%%%%%%%%%%
\section{A use case}

%%%%%%%%%%%%%%%%%%%%%%%%%%%%%%%%%%%%%%%%%%%%%%%%%%%%%%%%%%%%%%%%%%%%%%%%%%%
\section{Usage}

Figure \ref{fig-usage} contains the usage message given by the program.
In this section we explain some of the most important \starspan{} commands.
\begin{figure}[!ht]
\centering
\caption{Usage message}
\scriptsize{
\begin{verbatim}
starspan 0.9beta (Nov 13 2004 15:11:22)    UNDER DEVELOPMENT

USAGE:
  starspan <inputs/commands/options>...

   inputs:
      -raster <filenames>...
      -vector <filename>
      -speclib <filename>
      -update-csv <filename>
      -update-dbf <filename>

   commands:
      -raster_field <name>
      -raster_directory <directory>
      -csv <name>
      -envi <name>
      -envisl <name>
      -stats outfile.csv [avg|mode|stdev|min|max]...
      -calbase <link> <filename> [<stats>...]
      -report
      -dbf <name>
      -dump_geometries <filename>
      -mr <prefix>
      -jtstest <filename>

   options:
      -fields field1 field2 ... fieldn
      -pixprop <minimum-pixel-proportion>
      -noColRow
      -noXY
      -fid <FID>
      -skip_invalid_polys
      -progress [value]
      -RID_as_given
      -verbose
      -ppoly
      -in
      -srs <srs>
\end{verbatim}
}
\label{fig-usage}
\end{figure}

%%%%%%%%%%%%%%%%%%%%%%%%%%%%%%%%%%%%%%%%%%%%%%%%%%%%%%%%%%%%%%%%%%%%%%%%%%%
\subsection{Inputs and general options}

Raster data is given through the \verb|-raster| option:

	\verb|-raster| $R_1\ R_2\ \cdots\ R_n$

where $R_1, R_2, \ldots, R_n$ are raster datasets recognized by the GDAL library.

Vector data is given with the \verb|-vector| option:
	
	\verb|-vector| $V$

where $V$ is a vector datasource recognized by the OGR library.

There are some general options ...

	\verb|-fields| field$_1$ field$_2\ \cdots$ field$_n$
	
Only the specified fields will be transferred from vector file $V$
to the output. Use spaces to separate the field names.  Example:

   \verb|starspan -vector V -raster R -csv my.csv  -fields species foo|

will extract the fields with names `species' and `foo.' If no field from the
vector metadata is desired, just give the keyword \verb|-fields| with no
arguments. By default, all fields from vector will be extracted.

	\verb|-pixprop| value

Minimum proportion of pixel area in intersection so that the pixel is included.
A value in [0.0, 1.0] must be given.
Only used in intersections resulting in polygons. 
By default, the pixel proportion is 0.5. 



%%%%%%%%%%%%%%%%%%%%%%%%%%%%%%%%%%%%%%%%%%%%%%%%%%%%%%%%%%%%%%%%%%%%%%%%%%%
\subsection{Statistics}

Command \verb|-stats| is used to generate basic band-wise statistics for each
intersecting feature. Example:

	\verb|starspan -vector V -raster R -stats mystats.csv avg mode|

Creates \verb|mystats.csv|, a CSV (comma-separated-values) file containing
computed statistics for each FID. In the above example, codes \verb|avg| and
\verb|mode|
indicate to compute the average and mode statistics. Available statistics 
are shown in Table \ref{table-stats}.
\begin{table}[!ht]
\centering
\caption{Statistics}
\begin{tabular}{|l|l|}
\hline 
  code        & computes for each feature\\
\hline
\verb|avg  |  &   average of pixels in feature\\
\verb|mode |  &   Most frequent value\\
\verb|stdev|  &   sample standard deviation of pixels in feature\\
\verb|min  |  &   minimum value of pixels in feature\\
\verb|max  |  &   maximum value of pixels in feature\\
\hline
\end{tabular}
\label{table-stats}
\end{table}

Output:

    * mystats.csv : containing computed statistics.

First line in resulting \verb|myfile.csv| contain column headers:

    * FID: Feature ID
    * Names of attributes from the given input vector $V$ associated to FID. 
	If option \verb|-fields| is given, only the given fields will be extracted.
    * Band names. Each name takes the form $s$\_Band\#, where $s$
	is the code of the statistics, eg. \verb|stdev|, and \# is the band number 
	according to given rasters.

Subsequent lines contain corresponding values for each FID.

%%%%%%%%%%%%%%%%%%%%%%%%%%%%%%%%%%%%%%%%%%%%%%%%%%%%%%%%%%%%%%%%%%%%%%%%%%%
\section{Software requirements}

	In order to build \starspan{} the following libraries are required.
	
	\begin{itemize}
		\item GEOS, Geometry Engine \citep{geos}.
				This library is required to support geometry operations.
				
		\item GDAL - Geospatial Data Abstraction Library \citep{gdal}.
	\end{itemize}


\section{Conclusion and further work}

	Currently \starspan{} has been tested and used on Linux based operating
	environments. On the works is to make the tool publicly available and easy
	to install on different operating systems.


	For up-to-date documentation, including installation instructions and user
	manual, please visit: \url{http://starspan.casil.ucdavis.edu}

\section*{Acknowledments}

	GDAL, GEOS, 


	
%%%%%%%%%%%%%%%%%%%%%%%%%%%%%%%%%%%%%%%%%%%%%%%%%%%%%%%%%%%%%%%%%%%%%%	
% Bibliographic references with the natbib package:
% Parenthetical: \citep{Bai92} produces (Bailyn 1992).
% Textual: \citet{Bai95} produces Bailyn et al. (1995).
% An affix and part of a reference:
%   \citep[e.g.][Ch. 2]{Bar76}
%   produces (e.g. Barnes et al. 1976, Ch. 2).

\begin{thebibliography}{}

% \bibitem[Names(Year)]{label} or \bibitem[Names(Year)Long names]{label}.
% (\harvarditem{Name}{Year}{label} is also supported.)
% Text of bibliographic item

\bibitem[Ustin et al. (2004)]{ustin04}
	
	Susan L. Ustin, Emma C. Underwood, Melinda
	J. Mulitsch, Jonathan A. Greenberg, Shawn C. Kefauver, Michael L.
	Whiting, Carlos A. Rueda, Carlos M. Ramirez, George J. Scheer, and Lawrence
	F. Ross. 2004.
	Mapping Invasive Plant Species in the Sacramento- San Joaquin Delta Region
	Using Hyperspectral Imagery.
	

\bibitem[GEOS (2004)]{geos}
	
	\url{http://geos.refractions.net}	

\bibitem[GDAL (2004)]{gdal}
	
	\url{http://www.gdal.org}	

\end{thebibliography}

\end{document}

